%%%%%%%%%%%%%%%%%%%%%%%%%%%%%%%%%%%%%%%%%%%%%%%%%%%%%%%%%%%%
% Re-definitions for the PLT Scheme manual style
\typeout{XXXXXXXXXXXXXXXXXXXXXXXXXXXXXXXXXXXXXXXXXXXXXXXXX}
\renewcommand{\sectionNewpage}{\newpage}

\renewcommand{\preDoc}{\sloppy}

\renewcommand{\ChapRef}[2]{\SecRef{#1}{#2}}
\renewcommand{\SecRef}[2]{\S#1 ``#2''}
\renewcommand{\ChapRefUC}[2]{\SecRefUC{#1}{#2}}
\renewcommand{\SecRefUC}[2]{\SecRef{#1}{#2}}

% Definitions for mathbook

% \texttip{math}{tip} gives a tooltip in MathJax, in LaTeX we must ignore the tip
\newcommand{\texttip}[2]{#1}


% \newtheorem requires the amsthm package

% Documentation of amsthm: "types of structures which are normally associated with each theorem style"
% plain         Theorem, Lemma, Corollary, Proposition, Conjecture, Criterion, Algorithm
% definition    Definition, Condition, Problem, Example
% remark        Remark, Note, Notation, Claim, Summary, Acknowledgment, Case, Conclusion

% plain  is the default style (depends on the document class)
% typically italic body text
\theoremstyle{plain} 
\newtheorem{theorem}{Theorem}[section]          % theorem counter reset, when section is incremented
\newtheorem{lemma}[theorem]{Lemma}              % use the counter in [theorem] i.e. use shared counter
\newtheorem{proposition}[theorem]{Proposition}
\newtheorem{corollary}[theorem]{Corollary}
\newtheorem{exercise}[theorem]{Exercise}        % middle argument is the counter
\newtheorem{oevelse}[theorem]{Øvelse}            % middle argument is the counter

% definition style, typically roman text
\theoremstyle{definition}                       % 
\newtheorem{definition}{Definition}[section]
\newtheorem{conjecture}{Conjecture}[section]
\newtheorem{example}{Example}[section]
\newtheorem{eksempel}{Eksempel}[section]

% remark style, typically roman text
\theoremstyle{remark}
\newtheorem*{remark}{Remark}                   % a star means no numbering
\newtheorem*{note}{Note}
\newtheorem{case}{Case}
